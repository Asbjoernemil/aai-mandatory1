\documentclass[a4paper, twocolumn]{article}

\usepackage[utf8]{inputenc}
\usepackage[danish]{babel}
\usepackage[backend=biber]{biblatex}
\addbibresource{./refs.bib}

\usepackage{graphicx}
\usepackage{listings}
\usepackage{color}
\usepackage{booktabs}

\definecolor{lightgray}{gray}{0.9}

\lstset{
    showstringspaces=false,
    basicstyle=\ttfamily\small,
    keywordstyle=\color{blue},
    commentstyle=\color[gray]{0.6},
    stringstyle=\color[RGB]{255, 150, 75}
}

\author{Henrik Strøm}
\title{Forudsigelse af Brændstofeffektivitet i Biler:\\En Maskinlæringstilgang}

\begin{document}

\twocolumn[
    \begin{@twocolumnfalse}
        \maketitle
        \begin{abstract}
            Dette papir undersøger forudsigelse af brændstofeffektivitet i biler ved hjælp af både regressions- og klassifikationstilgange. Vi analyserer et datasæt med 398 biler med syv features (cylindre, displacement, hestekræfter, vægt, acceleration, modelår og origin) for at forudsige miles per gallon (MPG). Vores regressionsmodel opnår en R² på 0.847 og RMSE på 2.99 mpg på testdata, mens vores logistiske regressionsklassifikator opnår 87.3\% nøjagtighed i at forudsige om en bil er brændstofeffektiv. Analysen viser at vægt og displacement er de stærkeste negative prædiktorer for brændstofeffektivitet, mens modelår viser en positiv korrelation, hvilket indikerer forbedringer i bilteknologi over tid.
        \end{abstract}
    \end{@twocolumnfalse}
    \vspace{1cm}
]

\section{Introduktion}
\label{sec:intro}

Brændstofeffektivitet vil altid være aktuelt fra et miljømæssigt og økonomisk perspektiv. At forstå hvilke køretøjskarakteristika der stærkest påvirker brændstofforbrug kan informere både forbrugeres købsbeslutninger og prioriteter i bildesign.
 
Dette papir analyserer et datasæt med 398 køretøjer fra 1970'erne og begyndelsen af 1980'erne ved hjælp af både regressions- og klassifikationsmetoder for at forudsige brændstofeffektivitet (MPG).

\section{Forskningsspørgsmål}
\label{sec:research}

Denne undersøgelse adresserer to primære forskningsspørgsmål:

\textbf{Q1:} Kan vi nøjagtigt forudsige et køretøjs brændstofeffektivitet (MPG) baseret på dets tekniske specifikationer?

\textbf{Q2:} Kan vi pålideligt klassificere køretøjer som brændstofeffektive eller ej baseret på de samme karakteristika?

Disse spørgsmål er komplementære: Q1 giver præcise numeriske forudsigelser, mens Q2 tilbyder en simplere binær klassifikation der kan være mere praktisk i beslutningskontekster.

\section{Metoder}
\label{sec:methods}

\subsection{Datasæt}
Datasættet indeholder 398 biler med otte features: MPG (målvariabel), cylindre, displacement, hestekræfter, vægt, acceleration, modelår, origin og bilnavn. Seks observationer med manglende hestekræfter blev fjernet, hvilket resulterer i 392 komplette observationer.

\subsection{Regressionsproblem}
For regressionsopgaven forudsiger vi kontinuerte MPG-værdier. Datasættet blev opdelt i træning (70\%), validering (15\%) og test (15\%). Vi sammenlignede tre modeller: standard Lineær Regression, Ridge Regression med varierende alpha-værdier (0.1, 1.0, 10) og Lasso Regression (alpha=0.1).

Modelvalg var baseret på valideringssættets præstation ved brug af to metrikker: R² (determinationskoefficienten) og RMSE (Root Mean Squared Error). R² indikerer andelen af varians forklaret af modellen, mens RMSE giver den gennemsnitlige forudsigelsefejl i de oprindelige enheder (mpg).

\subsection{Klassifikationsproblem}
For klassifikation skabte vi en binær målvariabel ved at bruge median MPG (22.8) som tærskel: køretøjer over denne værdi klassificeres som "brændstofeffektive" (klasse 1), andre som "ikke brændstofeffektive" (klasse 0). Dette skaber et balanceret datasæt med 196 instanser i hver klasse.

Data blev opdelt 80\% træning og 20\% test. Vi anvendte StandardScaler til at normalisere features, da logistisk regression er følsom over for feature-skalaer. Hyperparameter-tuning blev udført ved hjælp af GridSearchCV med 5-fold cross-validation, hvor vi testede regulariseringsparameteren C = [0.01, 0.1, 1, 10, 100].

Modelpræstation blev evalueret ved brug af accuracy, precision, recall og F1-score. Derudover undersøgte vi ROC-kurven og AUC (Area Under Curve) for at vurdere modellens evne til at diskriminere mellem klasser på tværs af forskellige klassifikationstærskler.

\section{Analyse}
\label{sec:analysis}

\subsection{Regressionsanalyse}
Tabel \ref{tab:regression} viser valideringspræstation for alle testede modeller. Lasso Regression (alpha=0.1) opnåede den bedste præstation med R²=0.802 og RMSE=3.43 mpg.

\begin{table}[h]
\centering
\caption{Sammenligning af regressionsmodeller på valideringssæt}
\label{tab:regression}
\begin{tabular}{lcc}
\toprule
Model & R² & RMSE \\
\midrule
Lineær Regression & 0.790 & 3.53 \\
Ridge (α=0.1) & 0.790 & 3.53 \\
Ridge (α=1.0) & 0.792 & 3.52 \\
Ridge (α=10) & 0.798 & 3.46 \\
Lasso (α=0.1) & \textbf{0.802} & \textbf{3.43} \\
\bottomrule
\end{tabular}
\end{table}

Den valgte Lasso-model blev gendannet på de kombinerede trænings- og valideringsdata og derefter evalueret på det tilbageholdte testsæt. Endelig testpræstation: R²=0.847, RMSE=2.99 mpg, MAE=2.30 mpg.

Figur \ref{fig:regression} viser forudsagte versus faktiske MPG-værdier. Det stærke lineære forhold indikerer god modelpræstation, selvom der er nogen spredning særligt ved højere MPG-værdier.

\begin{figure}[h]
    \centering
    \includegraphics[width=0.48\textwidth]{figures/regression-results.png}
    \caption{Faktiske vs. forudsagte MPG på testsæt. Punkter tæt på den diagonale linje indikerer nøjagtige forudsigelser.}
    \label{fig:regression}
\end{figure}

\subsection{Klassifikationsanalyse}
GridSearchCV identificerede C=0.01 som optimal og opnåede 91.71\% cross-validation accuracy. På testsættet opnåede modellen 87.3\% accuracy med fremragende recall (97.3\%), men lavere precision (80.0\%).

Confusion matrix (Figur \ref{fig:confusion}) viser 36 true positives, 33 true negatives, 9 false positives og kun 1 false negative. Dette indikerer at modellen er konservativ og lejlighedsvis klassificerer ikke-effektive biler som effektive, men sjældent overser virkelig effektive køretøjer.

\begin{figure}[h]
    \centering
    \includegraphics[width=0.45\textwidth]{figures/confusion-matrix.png}
    \caption{Confusion matrix der viser klassifikationsresultater}
    \label{fig:confusion}
\end{figure}

ROC-kurve analysen gav en AUC på 0.955, hvilket indikerer fremragende diskriminerende evne på tværs af forskellige klassifikationstærskler.

\section{Fund}
\label{sec:findings}

\subsection{Regressionsfund}
Lasso-regressionsmodellen forudsiger succesfuldt brændstofeffektivitet med en gennemsnitlig fejl på 2.99 mpg (RMSE) på usete testdata og forklarer 84.7\% af variansen i MPG (R²=0.847). Feature importance-analyse afslører:

\begin{itemize}
\item \textbf{Negative prædiktorer:} Vægt (-0.358), cylindre (-0.333) og displacement (-0.320) reducerer brændstofeffektivitet mest
\item \textbf{Positive prædiktorer:} Modelår (+0.273) og origin (+0.219) forbedrer effektivitet
\item \textbf{Blandede effekter:} Hestekræfter viser negativ påvirkning (-0.266) mens acceleration har svag positiv effekt (+0.075)
\end{itemize}

Modellen generaliserer godt til nye data, med træningsnøjagtighed (91.7\%) tæt på testnøjagtighed (87.3\%), hvilket indikerer ingen overfitting.

\subsection{Klassifikationsfund}
Den logistiske regressionsklassifikator opnår 87.3\% accuracy i at skelne brændstofeffektive fra ikke-effektive køretøjer. Nøglemetrikker:

\begin{itemize}
\item Høj recall (97.3\%): Identificerer succesfuldt 97.3\% af virkelig effektive biler
\item God precision (80.0\%): Når den forudsiger "effektiv", korrekt 80\% af tiden
\item Fremragende AUC (0.955): Stærk diskriminerende evne
\end{itemize}

Modellen viser minimal overfitting, med trænings- (91.7\%) og test-accuracy (87.3\%) inden for acceptabel rækkevidde. Den enkelte false negative og ni false positives antyder en lidt konservativ klassifikator der foretrækker false positives frem for at overse effektive køretøjer.

\section{Konklusion}
\label{sec:conclusion}

Denne undersøgelse adresserer succesfuldt begge forskningsspørgsmål ved brug af det automotive datasæt.

\textbf{Q1 (Regression):} Ja, vi kan nøjagtigt forudsige køretøjs MPG med R²=0.847 og RMSE=2.99 mpg. Lasso-modellen med L1-regularisering (α=0.1) ydede bedre end både standard lineær regression og Ridge-alternativer. Køretøjsvægt fremstår som den stærkeste prediktor, hvor tungere biler konsekvent viser lavere brændstofeffektivitet.

\textbf{Q2 (Klassifikation):} Ja, vi kan pålideligt klassificere køretøjer som brændstofeffektive eller ej med 87.3\% accuracy og fremragende AUC (0.955). Den logistiske regressionsmodel med minimal regularisering (C=0.01) giver robuste binære forudsigelser velegnede til praktisk beslutningstagning.

De komplementære tilgange afslører konsistente mønstre: vægt, cylindre og displacement påvirker effektivitet negativt, mens nyere modelår viser forbedringer. Den stærke præstation af begge modeller validerer maskinlæringens anvendelighed til analyse af brændstofeffektivitet i biler.

Fremtidigt arbejde kunne udforske ikke-lineære relationer ved brug af polynomielle features eller træ-baserede metoder og udvide analysen til moderne køretøjsdata for at vurdere om historiske relationer forbliver gyldige med nuværende bilteknologi.

\end{document}